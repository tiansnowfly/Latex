\documentclass[12pt,a4paper]{article}
\usepackage{ctex}
\usepackage{amsmath,amscd,amsbsy,amssymb,latexsym,url,bm,amsthm}
\usepackage{epsfig,graphicx,subfigure}
\usepackage{enumitem,balance}
\usepackage{wrapfig}
\usepackage{mathrsfs,euscript}
\usepackage[usenames]{xcolor}
\usepackage{hyperref}
\usepackage{booktabs}
\usepackage[vlined,ruled,linesnumbered]{algorithm2e}
\hypersetup{colorlinks=true,linkcolor=black}

\newtheorem{theorem}{Theorem}
\newtheorem{lemma}[theorem]{Lemma}
\newtheorem{proposition}[theorem]{Proposition}
\newtheorem{corollary}[theorem]{Corollary}
\newtheorem{exercise}{Exercise}
\newtheorem*{solution}{Solution}
\newtheorem{definition}{Definition}
\theoremstyle{definition}

\renewcommand{\thefootnote}{\fnsymbol{footnote}}

\newcommand{\postscript}[2]
 {\setlength{\epsfxsize}{#2\hsize}
  \centerline{\epsfbox{#1}}}

\renewcommand{\baselinestretch}{1.0}

\setlength{\oddsidemargin}{-0.365in}
\setlength{\evensidemargin}{-0.365in}
\setlength{\topmargin}{-0.3in}
\setlength{\headheight}{0in}
\setlength{\headsep}{0in}
\setlength{\textheight}{10.1in}
\setlength{\textwidth}{7in}
\makeatletter \renewenvironment{proof}[1][Proof] {\par\pushQED{\qed}\normalfont\topsep6\p@\@plus6\p@\relax\trivlist\item[\hskip\labelsep\bfseries#1\@addpunct{.}]\ignorespaces}{\popQED\endtrivlist\@endpefalse} \makeatother
\makeatletter
\renewenvironment{solution}[1][Solution] {\par\pushQED{\qed}\normalfont\topsep6\p@\@plus6\p@\relax\trivlist\item[\hskip\labelsep\bfseries#1\@addpunct{.}]\ignorespaces}{\popQED\endtrivlist\@endpefalse} \makeatother

\begin{document}
\noindent

%========================================================================
\noindent\framebox[\linewidth]{\shortstack[c]{
\Large{\textbf{Lab01-Algorithm Analysis}}\vspace{1mm}\\
CS214-Algorithm and Complexity, Xiaofeng Gao, Spring 2019.}}
\begin{center}
\footnotesize{\color{red}$*$ If there is any problem, please contact TA Mingran Peng. Also please use English in homework.}

% Please write down your name, student id and email.
\footnotesize{\color{blue}$*$ Name:\quad \underline{��ѩ��}  \quad Student ID:\quad \underline{515030910347}\quad Email: \quad \underline{13487426939@qq.com}}
\end{center}
\begin{enumerate}
\item
\begin{solution}
The following is solution process:
\newline
\\
	\begin{tabular}{c|c| c }
		\toprule[2pt]
		\textbf{Algorithm} & \textbf{Time Complexity} & \textbf{Space Complexity} \\
		\hline
		\hline
		$InsertionSort$& $O(n^2)$, $\Omega(n)$  &  $\Theta(1)$  \\
		
		$CocktailSort$ & \color{blue}$O(n^2)$,$\Omega(n)$ &  \color{blue}$\Theta(1)$ \\

		$SelectionSort$ & \color{blue}$O(n^2)$,$\Omega(n)$ &  \color{blue}$\Theta(1)$ \\
		\bottomrule[2pt]


	\end{tabular}
\\
\\
\textbf{CocktailSort:}\newline
\textbf{Line 4:}\quad The number of \textbf{comparisons} carried out by Cocktailsort in loop
\begin{equation}
n-1
\end{equation}
and the number of \textbf{swap and assignment operation} at at least(\textbf{best})
\begin{equation}
0
\end{equation}
at most(\textbf{worst})
\begin{equation}
2(n-1)+2(n-3)+\dots+1
\end{equation}
\textbf{Line 9:}\quad The number of \textbf{comparisons} carried out by in loop at most least(\textbf{best})
\begin{equation}
n-2
\end{equation}
and the number of \textbf{swap and assignment operation} at least(\textbf{best})
\begin{equation}
0
\end{equation}
at most(\textbf{worst})
\begin{equation}
2(n-2)+2(n-4)+\dots+0
\end{equation}
\textbf{so,} the number of total executed in the loop of while at least\textbf{(best)}
\begin{equation}
T(n)=n-1+n-2=2n-3
\end{equation}
at most \textbf{(worst)}
\begin{equation}
T(n)=n-1+n-2+2\sum_{0}^{n-1}{k}=n^2+n-3
\end{equation}
\textbf{so,Time Complexity is $O(n^2)$ and $\Omega(n)$,and Space Complexity is $\Theta(1)$.}
\\
\\
\textbf{SelectionSort:}\newline
\textbf{Line $2$:} \quad The loop executed from $1$ to $n-1$\;\textbf{so},the number of \textbf{assignment operations} \\ carried out by SelectionSort are: \newline
\begin{equation}
2(n-1)
\end{equation}
\textbf{Line $4$:} \quad The comparisons of $A[j]>max$ will be executed when
\\
$j=2$,the number of comparisons are $n-1$;\\
$j=3$,the number of comparisons are $n-2$;\\
\vdots \\
$j=n$,the number of comparisons are $0$; \\
\textbf{so},the number of \textbf{comparisons} carried out by SelectionSort are:
\begin{equation}
\sum_0^{n-1}{j}=\frac{n(n-1)}{2}
\end{equation}
The number of \textbf{assignment operations} carried out in second loop are \\
at most:
\begin{equation}
n(n-1)
\end{equation}
at least:
\begin{equation}
0
\end{equation}
The number of total executed in SelectionSort are at most\textbf{(worst)}:
\begin{equation}
T(n)=2(n-1)+\frac{n(n-1)}{2}+n(n-1)=\frac{1}{2}(3n^2+n-4);
\end{equation}
The number of total executed in SelectionSort are at most\textbf{(best)}:
\begin{equation}
T(n)=2(n-1)+\frac{n(n-1)}{2}=\frac{1}{2}(n^2+3n-4);
\end{equation}
\textbf{so,Time Complexity is $O(n^2)$ and $\Omega(n)$,and Space Complexity is $\Theta(1)$.}
\end{solution}
\item
\begin{solution}
The following is solution process:\\
\textbf{(a) Two stack to simulated queue:}\\
\indent \hspace{0.8cm} $\bullet$ \textbf{stack1} is used for \textbf{enqueue},\textbf{stack2} is used for \textbf{dequeue};\\
\indent \hspace{0,8cm} $\bullet$ when pushing an element,we push it into \textbf{stack1};\\
\indent \hspace{0.8cm} $\bullet$ when popping an element,if \textbf{stack2} is empty,we push all elements in \textbf{stack1} \\ \indent \hspace{1.1cm} into \textbf{stack2},if \textbf{stack2} is not empty,we pop an element in \textbf{stack2} directly.\\
\textbf{(b) Time Complexity(potential function):}\\
\indent \hspace{1cm}\textbf{Potential function:} $\Phi(S)$ denote the num[i] of items in \textbf{stack1}.\\
\indent \hspace{1cm}\textbf{State:} Here state $S_i$ refers to the state of the  \textbf{stack1} after the $i$-th.\\
\indent \hspace{1cm}\textbf{Correctness:} $\Phi(S_i) \geq 0= \Phi(S_0)$ for any i;\\
\indent \hspace{1cm}According to the definition of $\Phi(S)$,we know
\begin{equation}
\Phi(S_i)=num[i]
\end{equation}
\indent \hspace{3cm} \textbf{PUSH:} $\hat{C_i}=C_i+\Phi(S_i)-\Phi(S_0)$\\
\indent \hspace{5.3cm}$=1+num[i]-num[i-1]$\\
\indent \hspace{5.3cm}$=1+num[i-1]+1-num[i-1]$\\
\indent \hspace{5.3cm}$=2$\\
\indent \hspace{3cm} \textbf{POP:} \\
\indent \hspace{4cm}$\bullet$ if \textbf{stack2} is not empty:\\
\indent \hspace{5cm} $\hat{C_i}=C_i+\Phi(S_i)-\Phi(S_0)$\\
\indent \hspace{5.5cm} $=1+num[i]-num[i-1]$\\
\indent \hspace{5.5cm} $=1$\\
\indent \hspace{4cm}$\bullet$ if \textbf{stack2} is empty:\\
\indent \hspace{5cm} $\hat{C_i}=C_i+\Phi(S_i)-\Phi(S_0)$\\
\indent \hspace{5.5cm} $=num[i-1]+1+(num[i]-num[i-1])$\\
\indent \hspace{5.5cm} $=num[i-1]+1+(0-num[i-1])$\\
\indent \hspace{5.5cm} $=1$\\
\textbf{Thus,} starting from two empty stacks,any sequence of \textbf{n1 PUSH},\textbf{n2 POP} \\ operations takes at most
\begin{equation}
T(n)=\sum_{i}^{n}{C_i}\leq \sum_{i}^{n}{\hat{C_i}}=2n_1+n_2. \quad Here \quad n=n_1+n_2  \quad \textbf{and}\quad n_1 \geq n_2.
\end{equation}
\end{solution}
\end{enumerate}
\end{document}
