\documentclass[12pt,a4paper]{article}
\usepackage{ctex}
\usepackage{amsmath,amscd,amsbsy,amssymb,latexsym,url,bm,amsthm}
\usepackage{epsfig,graphicx,subfigure}
\usepackage{enumitem,balance}
\usepackage{wrapfig}
\usepackage{mathrsfs,euscript}
\usepackage[usenames]{xcolor}
\usepackage{hyperref}
\usepackage{booktabs}
\usepackage[vlined,ruled,linesnumbered]{algorithm2e}
\usepackage{pstricks}
\usepackage[all]{xy}
\hypersetup{colorlinks=true,linkcolor=black}

\newtheorem{theorem}{Theorem}
\newtheorem{lemma}[theorem]{Lemma}
\newtheorem{proposition}[theorem]{Proposition}
\newtheorem{corollary}[theorem]{Corollary}
\newtheorem{exercise}{Exercise}
\newtheorem*{solution}{Solution}
\newtheorem{definition}{Definition}
\theoremstyle{definition}

\renewcommand{\thefootnote}{\fnsymbol{footnote}}

\newcommand{\postscript}[2]
 {\setlength{\epsfxsize}{#2\hsize}
  \centerline{\epsfbox{#1}}}

\renewcommand{\baselinestretch}{1.0}

\setlength{\oddsidemargin}{-0.365in}
\setlength{\evensidemargin}{-0.365in}
\setlength{\topmargin}{-0.3in}
\setlength{\headheight}{0in}
\setlength{\headsep}{0in}
\setlength{\textheight}{10.1in}
\setlength{\textwidth}{7in}
\makeatletter \renewenvironment{proof}[1][Proof] {\par\pushQED{\qed}\normalfont\topsep6\p@\@plus6\p@\relax\trivlist\item[\hskip\labelsep\bfseries#1\@addpunct{.}]\ignorespaces}{\popQED\endtrivlist\@endpefalse} \makeatother
\makeatletter
\renewenvironment{solution}[1][Solution] {\par\pushQED{\qed}\normalfont\topsep6\p@\@plus6\p@\relax\trivlist\item[\hskip\labelsep\bfseries#1\@addpunct{.}]\ignorespaces}{\popQED\endtrivlist\@endpefalse} \makeatother

\begin{document}
\noindent

%========================================================================
\noindent\framebox[\linewidth]{\shortstack[c]{
\Large{\textbf{Lab01-Algorithm Analysis}}\vspace{1mm}\\
CS214-Algorithm and Complexity, Xiaofeng Gao, Spring 2019.}}
\begin{center}
\footnotesize{\color{red}$*$ If there is any problem, please contact TA Mingran Peng. Also please use English in homework.}

% Please write down your name, student id and email.
\footnotesize{\color{blue}$*$ Name:\quad \underline{��ѩ��}  \quad Student ID:\quad \underline{515030910347}\quad Email: \quad \underline{13487426939@qq.com}}
\end{center}
\begin{enumerate}
\item
\begin{solution}
The following is solution process:\\
\textbf{(a)} \quad recurrence tree T(n):\\
\begin{flushleft}
\begin{displaymath}
\xymatrix{
&&&& \textbf{\large T(n)} &&&&&\textbf{\large computation}&\\
&&&& \blacksquare \ar[dl] \ar[dr] &&&&& O(n\log n) &\\
&&&\blacksquare \ar[dl] \ar[d] & & \blacksquare \ar[d] \ar[dr] &&&&2O(\frac{n}{2}\log \frac{n}{2})&\\
&&\blacksquare &\blacksquare && \blacksquare & \blacksquare &&&2^{j-1}O(\frac{n}{2^{j-1}}\log \frac{n}{2^{j-1}})\\
&&\textbf{\vdots} && \textbf{\vdots} && \textbf{\vdots} &&&\\
&\blacksquare \ar[dl] \ar[d] && \blacksquare \ar[dl] \ar[d]&&& \blacksquare \ar[dr] \ar[d] &&&\\
\blacksquare & \blacksquare & \blacksquare & \blacksquare & \textbf{\dots}& \textbf{\dots}& \blacksquare & \blacksquare &&2^{k-1}O(\frac{n}{2^{k-1}}\log \frac{n}{2^{k-1}})\\
}
\end{displaymath}\\
\textbf{Complexity of T(n)=sum up all computations at each level.}
\end{flushleft}
\textbf{(b) \quad No,because $O(n\log n)$ is not same as $O(n^d)$.}\\
\indent \quad \qquad \textbf{As we all know:$k=\log n$}\\
\begin{equation}
T(n)=\sum_0^{k-1}{2^{j-1}O(\frac{n}{2^{j-1}}\log \frac{n}{2^{j-1}})}\\
=\sum_1^{k}{2^{j}O(\frac{n}{2^{j}}\log \frac{n}{2^{j}})}\\
=\sum_1^{k}{O(n\log \frac{n}{2^{j}})}\\
\end{equation}
\begin{equation}
=\sum_1^{k}{O(n\log \frac{2^k}{2^{j}})}=\sum_1^{k}{O(n\log 2^j)}\\
=O(n\log^2 n)
\end{equation}
\end{solution}
\item
\begin{solution}
The following is solution process:\\
\textbf{(a)} The answer is in homework2.cpp\\
\textbf{(b)} recurrence T(n) by using Master Theorem:\\
\begin{equation}
T(n)=2T(\frac{n}{2})+O(n^2)=O(n^2)
\end{equation}

\end{solution}
\item
\begin{solution}
The following is solution process:\\
\textbf{(a)} \quad induction:\\
\indent \hspace{1cm} $\bullet$ n=1,we do not need sort:;\\
\indent \hspace{1cm} $\bullet$ n=2,obviously,we can use a comparator to sort two elements;\\
\indent \hspace{1cm} $\bullet$ we assume that we can sort k elements by using sort working:\\
\indent \hspace{1cm} $\bullet$ n=k+1;we can construct a function:
\begin{equation*}
    f(x)=
    \begin{cases}
    \ 0 & \text{if } x<a_{k+1} \\
    \ 1 & \text{if } x>=a_{k+1} \\
    \end{cases}
\end{equation*}
\indent \hspace{1.4cm} we can sort k elements to $\{0,0\dots,1,1\}$;\\
\indent \hspace{1cm} $a_{k+1}=1$,and we can compare $a_{k+1}$ with element in $\{0,0\dots,1,1\}$ one-by-one.\\
\indent \hspace{1cm} obviously,we can find a right index for $a_{k+1}$ after k comparisons.\\
\indent \hspace{1cm} \textbf{So,} a transposition network with n input is a sorting network if and only if it \\
\indent \hspace{1cm} sort the sequence $\{n,n-1\dots,1\}$.\\
\textbf{(b)} \quad The answer is in homework2.py\\
\end{solution}
\end{enumerate}
\end{document}
