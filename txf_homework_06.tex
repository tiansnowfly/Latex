\documentclass[12pt,a4paper]{article}
\usepackage{ctex}
\usepackage{amsmath,amscd,amsbsy,amssymb,latexsym,url,bm,amsthm}
\usepackage{epsfig,graphicx,subfigure}
\usepackage{enumitem,balance}
\usepackage{wrapfig}
\usepackage{mathrsfs,euscript}
\usepackage[usenames]{xcolor}
\usepackage{hyperref}
\usepackage[vlined,ruled,linesnumbered]{algorithm2e}
\hypersetup{colorlinks=true,linkcolor=black}

\newtheorem{theorem}{Theorem}
\newtheorem{lemma}[theorem]{Lemma}
\newtheorem{proposition}[theorem]{Proposition}
\newtheorem{corollary}[theorem]{Corollary}
\newtheorem{exercise}{Exercise}
\newtheorem*{solution}{Solution}
\newtheorem{definition}{Definition}
\theoremstyle{definition}

\renewcommand{\thefootnote}{\fnsymbol{footnote}}

\newcommand{\postscript}[2]
 {\setlength{\epsfxsize}{#2\hsize}
  \centerline{\epsfbox{#1}}}

\renewcommand{\baselinestretch}{1.0}

\setlength{\oddsidemargin}{-0.365in}
\setlength{\evensidemargin}{-0.365in}
\setlength{\topmargin}{-0.3in}
\setlength{\headheight}{0in}
\setlength{\headsep}{0in}
\setlength{\textheight}{10.5in}
\setlength{\textwidth}{7in}
\makeatletter \renewenvironment{proof}[1][Proof] {\par\pushQED{\qed}\normalfont\topsep6\p@\@plus6\p@\relax\trivlist\item[\hskip\labelsep\bfseries#1\@addpunct{.}]\ignorespaces}{\popQED\endtrivlist\@endpefalse} \makeatother
\makeatletter
\renewenvironment{solution}[1][Solution] {\par\pushQED{\qed}\normalfont\topsep6\p@\@plus6\p@\relax\trivlist\item[\hskip\labelsep\bfseries#1\@addpunct{.}]\ignorespaces}{\popQED\endtrivlist\@endpefalse} \makeatother

\begin{document}
\noindent

%========================================================================
\noindent\framebox[\linewidth]{\shortstack[c]{
\Large{\textbf{Lab06-Graph Exploration}}\vspace{1mm}\\
CS214-Algorithm and Complexity, Xiaofeng Gao, Spring 2019.}}


\begin{center}
\footnotesize{\color{red}$*$ If there is any problem, please contact TA Mingran Peng.}\par
% Please write down your name, student id and email.
\footnotesize{\color{blue}$*$ Name:\underline{��ѩ��}  \quad Student ID:\underline{515030910347} \quad Email: \underline{13487426939@qq.com}}
\end{center}
\begin{enumerate}

    \item
    \begin{solution}
    The following is the process of solution.\\
    \textbf{(a)}. Answer is 6 and the ccs.cpp program file is in the txf\_homework\_06 file.\\
    \textbf{(b)}. The ccs.gephi file is in the txf\_homework\_06 file.\\
    \end{solution}
\item
\begin{proof}
     The following is the process of proof.\\
     \indent \qquad If we use dfs to traverse a graph, that means we visit the vertex by using a stack.\\
     Proof by contradiction:\\
   \indent \qquad  Assume: if two intervals are not disjoint or one is contained within the other.\\
    and we visit u before v. we know : $PRE[u]<PRE[v]$ and $POST[u]<POST[v]$ \\
    but $POST[u]>PRE[v]$ and $POST[v]>POST[u]$
    According to the rule of stack. when we visited the u at first time,that means I should  push  the $PRE[u]$ into the stack and if there no other vertex connect to u,the we call
    $POST[u]$  when we visit u, then we pop vertex u out of stack. but in the case, we push
    $PRE[v] $ into stack before we call $POST[u]$ ;Then the stack exist $PRE[u]$ and $PRE[v]$, meanwhile $PRE[u]$ is on $PRE[v]$. so $PRE[u]$ must waiting for $PRE[v]$
    being popped out of stack, so we call the $POST[v] $  before $POST[u]$. so
    $POST[v]<POST[u]$, this is contradictory to $POST[v]>POST[u]$.\\
     \indent \qquad So,$\forall u, v \in V$, intervals $[PRE(u), POST(u)]$, $[PRE(v), POST(v)]$ are either disjoint or one is contained within the other.\par
\end{proof}
\item
\begin{solution}
  The following is the process of solution.\\
  In the case. we can change this question to the shortest path problem, and two computers communicate with a constant time t. So, we can use the BFS to solve this problem. The following is the pseudo code.\\
  \begin{algorithm}[H]
    \KwIn{computer s and computer t, and  an undirected Graph $G=(V,E)$ represent the relation of computer(connected(1) and unconnected(0)). vertex s $\epsilon$ \textbf{V}.}
    \KwOut{The shortest time needed to send message between two computers.}
    \caption{The shortest path.}
    \For{\textbf{each}  u $\epsilon$ \textbf{V}}{
           $DIST(u)=\infty;$\\
           $visited[i]=false;$\\
    }
    $DIST(s)=0;$\\
    $visited[s]=true;$(Mean the point is visited.)\\
    $Q.PUSH(s)$;(Using a queue Q to do BFS.)\\
    \While{Q is not empty}{
           $u=Q.POP()$;\\
           \For{\textbf{each} (u,v) $\epsilon$ \textbf{E}}{
                     \If {$DIST(v)=\infty$ \textbf{and} $visited=false$}{
                           $Q.PUSH(v)$;\\
                           $visited[s]=true$;\\
                           $DIST[v]=DIST(u)+\textbf{t}$;\\
                     }
                      \If{$v=\textbf{t}$}{
                            \Return{DIST[v]}\\
                     }
           }
    }
    \Return {unconnected}
  \end{algorithm}
  
  \textbf{Time complexity:} \\
  (1) Best case: if \textbf{s} connected with \textbf{v} or connected by a constant k vertex, then time complexity is o(1).\\
  (2)Worst case:if there is no path from \textbf{s} to \textbf{t}, then we must traverse all vertex and edges.then time complexity is o(V+E).\\
  (3) if two computers connect by k vertex, k=0,1,2,3\dots n-2; and probability is $\frac{1}{n-1}$. then time complexity is also o(V+E).\\
\end{solution}

\end{enumerate}

\vspace{20pt}

\textbf{Remark:} You need to include your .pdf and .tex files in your uploaded .rar or .zip file.

%========================================================================
\end{document}
