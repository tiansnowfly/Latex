\documentclass[12pt,a4paper]{article}
\usepackage{ctex}
\usepackage{amsmath,amscd,amsbsy,amssymb,latexsym,url,bm,amsthm}
\usepackage{epsfig,graphicx,subfigure}
\usepackage{enumitem,balance}
\usepackage{wrapfig}
\usepackage{mathrsfs,euscript}
\usepackage[usenames]{xcolor}
\usepackage{hyperref}
\usepackage[vlined,ruled,linesnumbered]{algorithm2e}
\usepackage{array}
\hypersetup{colorlinks=true,linkcolor=black}

\newtheorem{theorem}{Theorem}
\newtheorem{lemma}[theorem]{Lemma}
\newtheorem{proposition}[theorem]{Proposition}
\newtheorem{corollary}[theorem]{Corollary}
\newtheorem{exercise}{Exercise}
\newtheorem*{solution}{Solution}
\newtheorem{definition}{Definition}
\theoremstyle{definition}

\renewcommand{\thefootnote}{\fnsymbol{footnote}}

\newcommand{\postscript}[2]
 {\setlength{\epsfxsize}{#2\hsize}
  \centerline{\epsfbox{#1}}}

\renewcommand{\baselinestretch}{1.0}

\setlength{\oddsidemargin}{-0.365in}
\setlength{\evensidemargin}{-0.365in}
\setlength{\topmargin}{-0.3in}
\setlength{\headheight}{0in}
\setlength{\headsep}{0in}
\setlength{\textheight}{10.1in}
\setlength{\textwidth}{7in}
\makeatletter \renewenvironment{proof}[1][Proof] {\par\pushQED{\qed}\normalfont\topsep6\p@\@plus6\p@\relax\trivlist\item[\hskip\labelsep\bfseries#1\@addpunct{.}]\ignorespaces}{\popQED\endtrivlist\@endpefalse} \makeatother
\makeatletter
\renewenvironment{solution}[1][Solution] {\par\pushQED{\qed}\normalfont\topsep6\p@\@plus6\p@\relax\trivlist\item[\hskip\labelsep\bfseries#1\@addpunct{.}]\ignorespaces}{\popQED\endtrivlist\@endpefalse} \makeatother

\begin{document}
\noindent

%========================================================================
\noindent\framebox[\linewidth]{\shortstack[c]{
\Large{\textbf{Lab09-Approximation Algorithm}}\vspace{1mm}\\
CS214-Algorithm and Complexity, Xiaofeng Gao, Spring 2019.}}
\begin{center}
\footnotesize{\color{red}$*$ If there is any problem, please contact TA Jiahao Fan.}

% Please write down your name, student id and email.
\footnotesize{\color{blue}$*$ Name:\underline{��ѩ��} \quad Student ID:\underline{515030910347} \quad Email: \underline{13487426939@qq.com}}
\end{center}

\begin{enumerate}
    \item
        \begin{solution}
        The following is process of solution.\\
          \textbf{a}. The basic idea of a greedy algorithm is starting with an arbitrary center, and in each round, add the 'farthest'vertex to the center set until there are totaly k centers.\\
          \begin{algorithm}[H]
            \KwIn{an complete undirected graph with nonnegative edge G(V,E).}
            \KwOut{minimal $max_{v}\{cost(v,S)\}$.}
            \caption{Greedy approximation.}
            $\textbf{S } \leftarrow \{c_{1}\}$;  $c_1$ is an arbitrary center we choose at first .\\
            \For{ \textbf{v} $\in$  \textbf{V-S}}{
                    \If{$|S| \leq k$}{
                     Find the farthest vertex $v_i$ to center set.\\
                     $\textbf{S} \leftarrow \textbf{S} \cup \{c_i\}$.\\
                     }
            }
            \Return{\textbf{S}.}
          \end{algorithm}
          \textbf{b}. Let's consider a set  \textbf{S} and  two vertices $v_g$ and $v_opt$ , The optimal choice is to choose $v_{opt}$ but we choose the $v_g$ by greedy algorithm. In this case,  we can use an illustration to describe this question, then we can construct the following relation: \\
          \begin{equation*}
          \begin{cases}
            cost(v_g,S)=max\{s_1,s\}=s_{m1}; \\
            cost(v_{opt},S)=mas\{s_2,s\}=s_{m2};
            \end{cases}
          \end{equation*}
          \begin{figure}[htbp]
          \begin{minipage}[h]{0.8\textwidth}
          \centering
         \includegraphics[width=0.9\textwidth]{1.png}
         \caption{The flow of two nodes is 1}
         \end{minipage}
          \end{figure}
          According to the triangle inequality: \\
          $cost(v_g,S)=max\{s_1,s\}=s_{m1} \leq s_{2}+s \leq 2* max\{s_{2},s\} =2 cost(v_{opt},S) $; \\ then $cost(v_g,S) \leq 2 cost(v_{opt},S)$ ,so $\frac{cost(v_g,S)}{cost(v_{opt},S)} \leq 2$.\\
        \end{solution}
        \vspace{2cm}

    \item
    \begin{solution}
      The following is the process of proof.\\
      \textbf{proof:} In this case, we can use a vertex \textbf{v} which belongs to \textbf{R} , set \textbf{R-{v}} and set \textbf{S}, the goal is to find minimum cost tree, we can use an illustration to describe this question:\\
      \begin{figure}[htbp]
     \begin{minipage}[h]{0.8\textwidth}
     \centering
     \includegraphics[width=0.9\textwidth]{2.png}
     \caption{The flow of two nodes is 1}
    \end{minipage}
    \end{figure}
    \\
    we consider the a vertex \textbf{v} in \textbf{R},according to approximation, we know vertex \textbf{v} must connect to one vertex in \textbf{R-{v}}, but in optimal answer,we know \textbf{v} can connect to one vertex in \textbf{S}, \\ so, we can construct to the following relation by illustration above.\\
    \begin{equation*}
      \begin{cases}
        c_g=s_1+min\{s_2,s_3\};\\
        c_{opt}=s_2+s_3;\\
      \end{cases}
    \end{equation*}
    According to the triangle inequality: $c_g=s_1+min\{s_2,s_3\} \leq 2(s_2+s_3) \leq 2 c_{opt}$; For all vertices in R,we have same relation,so $C_g \leq 2 C_{opt}$, then $\frac{C_g}{C_{opt}}\leq 2.$\\
    \end{solution}

    \item
    \textbf{Minimum Weighted Vertex Cover:} \\
        \textbf{a. }Denote: we assume C is a cover vertex set, In graph G(V,E), for each vertex v $\in$ V, if v $\in$ C, $x(v)=1$ (vertex v in vertex cover set), if v $\notin$  C, $x(v)=0$ (vertex v  not in vertex cover set).at least one vertex is in vertex cover set for every edge:$x(v_j)+x(v_i) \geq 1;$\\
        \textbf{integer linear program:}\\
        \begin{equation*}
          \begin{cases}
            \sum_{v_i \in V} c_ix(v_i);\\
            x(v_j)+x(v_i) \geq 1;\\
            x(v_i),(v_j)={0,1};\\
            i \in \{1,2 \dots n\}
          \end{cases}
        \end{equation*}

        \textbf{b.  proof:} As we know, the following is an approximation algorithm with value  $m_{LP}(G)$ , we assume: $m^*{LP}$ is the best solution for linear program. Then $m^*(LP) \leq m^*(G)$;\\
        In solution \textbf{a}, we know  $x(v_j)+x(v_i) \geq 1$; \\ 
        so there must have one vertex which $x(v) \geq \frac{1}{2}$; $\{v_i,v_j\} \geq \frac{1}{2}$;\\
          $  m^*(LP) =\sum_{v_i \in V} c_ix(v_i) \geq \sum_{v_i \in V and x(v)\geq \frac{1}{2}}
            c_ix(v_i)$
            $\geq \sum_{v_i \in V and x(v)\geq \frac{1}{2}}
            \frac{1}{2}*c_i=\frac{1}{2} m_{LP}(G)$\\
            Then, $\Rightarrow  \frac{1}{2} m_{LP}(G) \leq m^*(G)$;\\
            So ,$ \Rightarrow m_{LP}(G)/m^*(G) \leq 2$.\\
    \item
    Give the corresponding $(I,sol,m,goal)$ for \textbf{Metric $k$-center} and \textbf{Minimum Weighted Vertex Cover} respectively.
    \textbf{Metric $k$-center}(I,sol,m,goal) :\\
         I=\{G(V,E)$|$G is a graph\}; poly time decidable.\\
         sol=\{U $\subseteq V| $arbitrary select a vertex as center, add the 'farthest' vertex to the center set\\  \indent \qquad until there are totaly k centers.\} \\
         m=farthest distance v to k-center set U; $v \not in U;$\\
         $goal =min\{m\};$\\
     \textbf{Minimum Weighted Vertex Cover} (I,sol,m,goal):\\
           I=\{G(V,E)$|$G is a graph\};poly time decidable.\\
          $sol={U \subseteq V| \forall (v,u) \in E (v \in V or u \in V)}.$\\
          $m=\sum_{v \in V} c(v)x(v). v \in V,x(v)=1;v \notin V, x(v)=0; $c(v) is the cost of vertex v.\\
          $goal=min\{m\}$\\ 
\end{enumerate}

\vspace{20pt}

\textbf{Remark:} You need to include your .pdf and .tex files in your uploaded .zip file.

%========================================================================
\end{document}
