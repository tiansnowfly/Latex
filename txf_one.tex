\documentclass[12pt,a4paper]{article}
\usepackage{ctex}
\usepackage{amsmath,amscd,amsbsy,amssymb,latexsym,url,bm,amsthm}
\usepackage{epsfig,graphicx,subfigure}
\usepackage{enumitem,balance}
\usepackage{wrapfig}
\usepackage{mathrsfs,euscript}
\usepackage[usenames]{xcolor}
\usepackage{hyperref}
\usepackage[vlined,ruled,linesnumbered]{algorithm2e}
\hypersetup{colorlinks=true,linkcolor=black}

\newtheorem{theorem}{Theorem}
\newtheorem{lemma}[theorem]{Lemma}
\newtheorem{proposition}[theorem]{Proposition}
\newtheorem{corollary}[theorem]{Corollary}
\newtheorem{exercise}{Exercise}
\newtheorem*{solution}{Solution}
\newtheorem{definition}{Definition}
\theoremstyle{definition}

\renewcommand{\thefootnote}{\fnsymbol{footnote}}

\newcommand{\postscript}[2]
 {\setlength{\epsfxsize}{#2\hsize}
  \centerline{\epsfbox{#1}}}

\renewcommand{\baselinestretch}{1.0}

\setlength{\oddsidemargin}{-0.365in}
\setlength{\evensidemargin}{-0.365in}
\setlength{\topmargin}{-0.3in}
\setlength{\headheight}{0in}
\setlength{\headsep}{0in}
\setlength{\textheight}{10.1in}
\setlength{\textwidth}{7in}
\makeatletter \renewenvironment{proof}[1][Proof] {\par\pushQED{\qed}\normalfont\topsep6\p@\@plus6\p@\relax\trivlist\item[\hskip\labelsep\bfseries#1\@addpunct{.}]\ignorespaces}{\popQED\endtrivlist\@endpefalse} \makeatother
\makeatletter
\renewenvironment{solution}[1][Solution] {\par\pushQED{\qed}\normalfont\topsep6\p@\@plus6\p@\relax\trivlist\item[\hskip\labelsep\bfseries#1\@addpunct{.}]\ignorespaces}{\popQED\endtrivlist\@endpefalse} \makeatother

\begin{document}
\noindent

%========================================================================
\noindent\framebox[\linewidth]{\shortstack[c]{
\Large{\textbf{Lab00-Proof}}\vspace{1mm}\\
CS214-Algorithm and Complexity, Xiaofeng Gao, Spring 2019.}}
\begin{center}
\footnotesize{\color{red}$*$ If there is any problem, please contact TA Jiahao Fan.}

% Please write down your name, student id and email.
\footnotesize{\color{blue}$*$ Name:\quad \underline{��ѩ��}  \quad Student ID:\quad \underline{515030910347} \quad Email:\quad \underline{13487426939@qq.com}}
\end{center}

\begin{enumerate}
    \item
    Prove that for any integer $n>2$, there is a prime $p$ satisfying $n<p<n!$. {\color{blue}(Hint: consider a prime factor $p$ of $n!-1$ and prove by contradiction)}
    \begin{proof}
    The following is proof process:
    \color{blue}
    \newline
        \quad As we all know,two adjacent numbers have no common divisor,so \textbf{n!} and \textbf{n!-1} have no common
        divisor.Meanwhile \textbf{1} to \textbf{n} are the common divisor of \textbf{n!},so they are not the
        common divisor of \textbf{n!-1}.
        \begin{itemize}
        \item If \textbf{n!-1} is prime:\\ $p=n!-1$. so
             $n<p<n!$.\textbf{so} the statement is \textbf{true}.
        \end{itemize}
        \begin{itemize}
        \item If \textbf{n!-1} is composite number:\\ we assume that \textbf{p} is a prime and a common divisor of \textbf{n!-1}, so $n<p<n!-1<n!$.\\ \textbf{so}
        the statement is \textbf{true}.\\
        \end{itemize}
        \BlankLine
        End of proof.
    \end{proof}

    \item
    Use the minimal counterexample principle to prove that for any integer $n>17$, there exist integers $i_n\ge 0$ and $j_n\ge 0$, such that $n = i_n \times 4 + j_n \times 7$.
    \begin{proof}
    The following is proof process:\\
    \color{blue}
    \textbf{when}\quad $n=18$\; \textbf{push}\quad $n=1 \times 4 + 2 \times 7;$\;\quad \textbf{so} statement is true; \\
    \quad \textbf{when}\quad $n=19$\; \textbf{push}\quad $n=3 \times 4 +1\times 7;$\;\quad \textbf{so} statement is true;\\
    \quad \textbf{when}\quad $n=20$\; \textbf{push}\quad $n=5 \times 4 +0\times 7;$\;\quad \textbf{so} statement is true;\\
    \quad \textbf{when}\quad $n=21$\; \textbf{push}\quad $n=0 \times 4 +3\times 7;$\;\quad \textbf{so} statement is true;\\
    \quad \textbf{when}\quad $n=22$\; \textbf{push}\quad $n=2 \times 4 +2\times 7;$\;\quad \textbf{so} statement is true;\\
    \quad \textbf{when}\quad $n=23$\; \textbf{push}\quad $n=4 \times 4 +1\times 7;$\;\quad \textbf{so} statement is true;\\
    \quad \textbf{when}\quad $n=24$\; \textbf{push}\quad $n=6 \times 4 +0\times 7;$\;\quad \textbf{so} statement is true;\\
    \quad \textbf{when}\quad $n \ge 25;n=N+7k,N=18,19,20,
    21,22,23,24,k=1,2,3 \dots;$\\
    \textbf{so},\quad for any integer $n \ge 18,$there exist
    integers $i_n \ge 0$ and $j_n \ge 0,$ such that $n = i_n \times 4 + j_n \times 7.$\\
    End of proof.
    \end{proof}

    \item
    Suppose $a_0=1$, $a_1=2$, $a_2=3$, and $a_k=a_{k-1}+a_{k-2}+a_{k-3}$ for $k \ge 3$. Use the strong principle of mathematical induction to prove that $a_n \le 2^n$ for any integer $n\ge 0$.
    \begin{proof}
    The following is proof process:
    \newline
        \color{blue}
        \textbf{Firstly:}\quad According to statement:\newline \indent \quad \textbf{If} $n=0,a_0=1 \leq 2^0,$
        $n=1,a_1=2 \leq 2^1,$
        $n=2,a_2=3 \leq 2^2;$ \quad \textbf{true}\\
        \indent \quad \textbf{If} $n=3,a_3=a_2+a_1+a_0=6<2^3;$\quad \textbf{true.}\\
        \textbf{Secondly:}\quad we assume that the inequality is true for any $3 \leq k \leq n,k=n,a_{n} \leq 2^n.$ \\
        \textbf{Thirdly:}\quad $$a_{n+1}=a_{n}+a_{n-1}+a_{n-2}
        \leq 2^n+2^{n-1}+2^{n-2} \leq 2^{n+1}.$$
        \textbf{Finally:} \quad For any integer $n \leq 0,a_n \leq 2^n.$\\
        End of proof.
    \end{proof}

    \item
    Prove, by mathematical induction, that
    $$
    (n+1)^2+(n+2)^2+(n+3)^2+\cdots +(2n)^2=\dfrac{n(2n+1)(7n+1)}{6}
    $$
    is true for any integer $n\ge 1$.
   \begin{proof}
   The following statement is proof process:\\
   \color{blue}For any integer $\textbf{k} \ge 2:$
   $$
   k^2=\dfrac{1}{3}{\big[ k(k+1)(k+2)-(k-1)k(k+1)\big]}-k\;
   $$
   So we can know:
   $$(n+1)^2+(n+2)^2+(n+3)^2+\cdots+(2n)^2$$
   $$\hspace{1.5cm}=\sum_{k=n+1}^{2n}{\dfrac{1}{3}\big[k(k+1)(k+2)
   -(k-1)k(k+1)\big]}-\sum_{k=n+1}^{2n}{k}$$
   $$\hspace{1.9cm}=\dfrac{1}{3}{\big[2n(2n+1)(2n+2)-n(n+1)(n+2)
   \big]}-\dfrac{1}{2}{n(3n+1)}$$
   $$\hspace{-4.7cm}=\dfrac{n(2n+1)(7n+1)}{6}$$
   End of proof.
   \end{proof}

\end{enumerate}

\vspace{20pt}

\textbf{Remark:} You need to include your .pdf and .tex files in your uploaded .rar or .zip file.

%========================================================================
\end{document}
