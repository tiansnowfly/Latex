\documentclass[12pt,a4paper]{article}
\usepackage{ctex}
\usepackage{amsmath,amscd,amsbsy,amssymb,latexsym,url,bm,amsthm}
\usepackage{epsfig,graphicx,subfigure}
\usepackage{enumitem,balance}
\usepackage{wrapfig}
\usepackage{mathrsfs,euscript}
\usepackage[usenames]{xcolor}
\usepackage{hyperref}
\usepackage[vlined,ruled,linesnumbered]{algorithm2e}
\hypersetup{colorlinks=true,linkcolor=black}

\newtheorem{theorem}{Theorem}
\newtheorem{lemma}[theorem]{Lemma}
\newtheorem{proposition}[theorem]{Proposition}
\newtheorem{corollary}[theorem]{Corollary}
\newtheorem{exercise}{Exercise}
\newtheorem*{solution}{Solution}
\newtheorem{definition}{Definition}
\theoremstyle{definition}

\renewcommand{\thefootnote}{\fnsymbol{footnote}}

\newcommand{\postscript}[2]
 {\setlength{\epsfxsize}{#2\hsize}
  \centerline{\epsfbox{#1}}}

\renewcommand{\baselinestretch}{1.0}

\setlength{\oddsidemargin}{-0.365in}
\setlength{\evensidemargin}{-0.365in}
\setlength{\topmargin}{-0.3in}
\setlength{\headheight}{0in}
\setlength{\headsep}{0in}
\setlength{\textheight}{10.1in}
\setlength{\textwidth}{7in}
\makeatletter \renewenvironment{proof}[1][Proof] {\par\pushQED{\qed}\normalfont\topsep6\p@\@plus6\p@\relax\trivlist\item[\hskip\labelsep\bfseries#1\@addpunct{.}]\ignorespaces}{\popQED\endtrivlist\@endpefalse} \makeatother
\makeatletter
\renewenvironment{solution}[1][Solution] {\par\pushQED{\qed}\normalfont\topsep6\p@\@plus6\p@\relax\trivlist\item[\hskip\labelsep\bfseries#1\@addpunct{.}]\ignorespaces}{\popQED\endtrivlist\@endpefalse} \makeatother

\begin{document}
\noindent

%========================================================================
\noindent\framebox[\linewidth]{\shortstack[c]{
\Large{\textbf{Lab10-Approximation \& Randomized Algorithm}}\vspace{1mm}\\
CS214-Algorithm and Complexity, Xiaofeng Gao, Spring 2019.}}


\begin{center}
\footnotesize{\color{red}$*$ If there is any problem, please contact TA Mingran Peng.}\par
% Please write down your name, student id and email.
\footnotesize{\color{blue}$*$ Name:\underline{��ѩ��}  \quad Student ID:\underline{515030910347} \quad Email: \underline{13487426939@qq.com}}
\end{center}
\begin{enumerate}
\item
\begin{enumerate}
\begin{solution}
  The following is the process of solution.\\
  \textbf{a.} For each $x_i, x_i \in \{0,1\}$, and for the convenience of using formula by integer programming, we use $\overline{x_i}=1-x_i$.\\
  \begin{equation*}
    \begin{cases}
      min\sum_{1}^{n}{x_i}\\
      \sum {\overline{x_j}}+\sum {x_i} \geq 1; x_i,x_j \in clause_1\\
      \sum {\overline{x_j}}+\sum {x_i} \geq 1; x_i,x_j \in clause_2\\
      \vdots \\
       \sum {\overline{x_j}}+\sum {x_i} \geq 1; x_i,x_j \in clause_m \\
       x_1,x_2,\dots x_n \in \{0,1\}\\
       \overline{x_1},\overline{x_2},\dots \overline{x_n} \in \{0,1\}\\
    \end{cases}
  \end{equation*}
  Then we can change the formula to:\\
  \begin{equation*}
    \begin{cases}
      min\sum_{1}^{n}{x_i}\\
      \sum {(1-x_j)}+\sum {x_i} \geq 1; x_i,x_j \in clause_1 \quad  i,j \in {1,2 \dots n}\\
      \sum {(1-x_j)}+\sum {x_i} \geq 1; x_i,x_j \in clause_2 \quad  i,j \in {1,2 \dots n}\\
      \vdots \\
       \sum {(1-x_j)}+\sum {x_i} \geq 1; x_i,x_j \in clause_m \quad  i,j \in {1,2 \dots n}\\
       x_1,x_2,\dots x_n \in \{0,1\}\\
    \end{cases}
  \end{equation*}
  \textbf{b.}LP-relaxation.\\
\begin{algorithm}[H]
  \KwIn{A CNF $\Phi$ with n boolean variables $\{x_i\}_{i=1}^n$ and m clauses with each clause consisting of 3 boolean variables.}
  \KwOut{Feasible satisfiable with fewest true boolean variables.}
  \caption{Approximation algorithm.}
  Find an optimal algorithm solution to the LP-relaxation.\\
  $count=0$;\\
  /For{i to n} {
       \If{$x_i \geq \frac{1}{3}$}{
            $round  x_i=1$;\\
            $count=count+1$;\\
       }
       \Else{
           $round x_i=0$;\\
       }
  }
  \Return{count}
\end{algorithm}
\end{solution}
$OPT_{LP} \leq 3* OPL_{ILP} \leq 3*OPL$\\
So, approximation algorithm ratio is 3.\\
\vspace{3cm}

\end{enumerate}
%    \begin{proof}
%        Uncomment this block to write your proof.
%    \end{proof}
\item

\begin{enumerate}
\item
\begin{solution}
  Random choose a position in $[l,r]$ , and choose to color in this position.Then we check the every position and every color for query. in this case,the pre-processing complexity is $O(n)$.\\
  We assume: the number of position of $i_{th}$ color pearl are $k_i$; Repeat the pre-processing 10 times. For query, it will be execute $10(r-l)$ times, since r-l can achieve n-1,so time complexity per query is $O(n)$.No extra space needed.Then, prove accuracy will be better than 99.9\%.\\
  \textbf{proof}( two cases):\\
  $\bullet$ For all color in $[l,r]$,$k_i$ is smaller than $\frac{1}{2}(r-l)$. after 10 operations, we can not find any color which satisfies the condition. So,the answer is there does not exist this color. This answer is true.Accuracy is 100\%.\\
  $\bullet$ For all color in $[l,r]$, if there exist one color: $k_i \geq \frac{1}{2}(r-l)+1$; when we choose a color in $[l,r]$ and repeat 10 times, the probability of $i_th$ color  not being chosen is $(1-\frac{k_i}{m})^{10}$.
  and $p=(1-\frac{k_i}{m})^{10} \leq (1-\frac{1}{2})^{10} \leq \frac{1}{2^{10}} \leq 0.1\%$.then.the probability that the answer will go wrong is not smaller than 0.1\%, So ,Accuracy is 99.9\%.\\
  So, the accuracy of this random algorithm will be better than 99.9\%.\\
\end{solution}
\end{enumerate}

%    \begin{proof}
%        Uncomment this block to write your proof.
%    \end{proof}




\end{enumerate}

\vspace{20pt}

\textbf{Remark:} You need to include your .pdf and .tex files in your uploaded .rar or .zip file.

%========================================================================
\end{document}
